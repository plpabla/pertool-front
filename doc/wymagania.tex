\section{Wymagania}
Wymagania projektowe zostaną zebrane w formie mapy historyjek użytkownika.

Kluczowe funkcjonalności aplikacji to
\begin{itemize}
	\item obsługa widoku strony www
	\item tworzenie diagramów
	\item analiza diagramów (np. dokonywanie obliczeń w celu wyznaczenia ścieżki krytycznej)
	\item generacja dodatkowych widoków (np. diagram Gantta)
	\item import i eksport diagramów
	\item obsługa sesji użytkownika (uwierzytelnianie, autoryzacja, zapisywanie diagramów na serwerze)
\end{itemize}
	
Każda z historyjek użytkownika powinna być związana z jedną z tych funkcjonalności. 

Każda z historyjek ma przypisanego aktora. Zidentyfikowano następujące role:
\begin{itemize}
	\item programista - przygotowuje aplikację
	\item autor - buduje diagram
	\item odbiorca - przegląda diagram
	\item użytkownik = autor+odbiorca
	\item właściciel aplikacji - posiada aplikację (może np zechcieć wyświetlać reklamy w celach zarobkowych, czy zbierać dane statystyczne) 
\end{itemize}

\subsection{Historyjki użytkownika tworzące ,,chodzący szkielet''}
Jest to minimalna funkcjonalność aplikacji wymagana, żeby uznać ją za użyteczną
\begin{itemize}
	\item jako programista chcę mieć zdefiniowany stos technologiczny, żebym wiedział w czym działać
	\item jako użytkownik chcę mieć dostępny widok strony podzielony na logiczne sekcje, żebym mógł korzystać ze wszystkich funkcjonalności
	\item jako programista chcę wiedzieć jak wygląda wysokopoziomowa architektura, żebym wiedział w jaki sposób zbudować system
	\item jako autor chce mieć dostępny toolbox żebym mógł wybrać jaką akcję chcę wykonać
	\item jako autor chcę rozpoczynać pracę od pustego obszaru roboczego jedynie z początkowym milestonem
	\item jako autor chcę móc dodać milestone do obszaru roboczego żebym mógł zacząć tworzyć plan projektu
	\item jako autor chcę móc połączyć dwa milestone'y linią reprezentującą wymaganą pracę, żebym mógł określić zależności czasowe
	\item jako autor chcę mieć możliwość przesuwania milestone'ów, żebym mógł przearanżować wygląd diagramu
	\item jako autor chcę móc dodać opis milestone'a, żebym wiedział czego on dotyczy
	\item jako autor chcę móc edytować nazwę milestone'a żebym mógł poprawić źle wprowadzoną nazwę
	\item jako autor chcę móc dodać złożoność czasową danego zadania, żebym mógł obliczyć kiedy kolejny milestone może zostać osiągnięty
	\item jako autor chcę móc edytować złożoność czasową, żeby dostosować plan do aktualnej sytuacji
	\item jako autor chcę móc dodać puste zależności, żebym mógł podkreślić zależności pomiędzy milestone'ami
	\item jako odbiorca chcę widzieć obliczone czasy osiągnięcia poszczególnych milestone'ów, żebym wiedział, czy plan projektu jest zgodny z celami biznesowymi
\end{itemize}